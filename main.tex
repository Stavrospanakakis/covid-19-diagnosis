\documentclass{article}

\usepackage{graphicx}
\usepackage{float}
\usepackage{multicol}
\usepackage{geometry}
 \geometry{
 a4paper,
 total={170mm,257mm},
 left=20mm,
 top=20mm,
 }


\title{Covid-19 study with Neural Networks}
\author{Stavros Panakakis, Sophia Tsivoula}

\begin{document}
	\maketitle

\section{Summary}		

\section{Introduction}
Coronavirus (COVID-19) is a newly infectious disease that has evolved into a pandemic in March 2020. Specifically, the first cases of Coronavirus (COVID-19) were detected in the city of Wuhan, China in December 2019 in a seafood wholesale market. A number of patients initially diagnosed with a form of pneumonia, that later discovered through samples of sufferers, that originates from an unknown, till then, beta-coronavirus virus \cite{firstcases}.

The most common symptoms of the virus today are fever, cough, fatigue, expectoration and shortness of breath. More rarely, sufferers experience headaches or dizziness, diarrhea, nausea and vomiting \cite{symptoms1}. Examining the samples from the sufferers, it was observed that some social groups are more likely to get infected and experience severe respiratory problems [Figure: \ref{fig:CovidCasesChina}] originated from the virus. \cite{patients}. 

\begin{figure}[H]
\includegraphics[width=\linewidth]{assets/cases-in-china.png}
  \caption{Impact of cardiovascular metabolic diseases on COVID-19 in China}
  \label{fig:CovidCasesChina}
\end{figure}

Coronavirus is a RNA positive-strand virus, thus has higher mutation rates than DNA virus. This is the reason why  Coronaviruses are easy to adopt in different environments in order to survive and reproduce. Although COVID-19 is not as virulent as influenza viruses, humans have developed a certain degree of immunity  through years of exposure, which did not happen with COVID-19, therefore making it more life threatening. On the other hand, the patients’ data is relatively encouraging since 85\% of COVID-19 patients suffered from mild infection, 10\% from severe and only 5\% of patients suffered from critical infection. Most critical COVID-19 cases are elderly people, people suffering from other diseases and individuals who have a weak immune system \cite{medicalCOVID}. This means that  COVID-19 is able to spread and reproduce easier and faster on the respiratory system of the patients and cause COVID-19 pneumonia [Figure: \ref{fig:CovidCasesChina}].

\begin{figure}[H]
\includegraphics[width=\linewidth]{assets/covid-vs-pneumonia-vs-normal.png}
  \caption{Healthy person vs Normal Pneumonia vs COVID-19 Pneumonia}
  \label{fig:xraycomparison}
\end{figure}


\section{Impact of Covid-19}

\subsection{Cases}

\begin{figure}[H]
\includegraphics[width=\linewidth]{cases/worldwide/plot.png}
  \caption{Covid-19 worldwide cases.}
  \label{fig:worldwidecases}
\end{figure}

\begin{figure}[H]
\includegraphics[width=\linewidth]{cases/greece/plot.png}
  \caption{Covid-19 Cases in Greece.}
  \label{fig:greececases}
\end{figure}

\subsection{Cost}

\section{Need for analytics}

\section{Review}
\textbf{Application of deep learning technique to manage COVID-19 in routine clinical practice using CT images: Results of 10 convolutional neural networks}\cite {firstreview}  
\\
\\
The first study focuses on the improvement of COVID-19 diagnosis process  and proposes artificial intelligent techniques for reliable and faster results from previous methods, such as computed tomography (CT). More specifically the study uses ten Convolutional Neural Networks (CNN) and explains the accuracy on each of them [Table: \ref{tab:table1}]. The network ResNet-101 \cite{resnet} achieved 99.51\% accuracy, the best one described in this study. In the study participated 108 patients (48 female and 60 male) positive with COVID-19 and 86 (35 female and 51 male)  non-COVID-19 patients. The age of COVID-19 positive is 50.22 ± 10.85 and of non-COVID-19 is 61.45 ± 15.04. In order to create the CNN, the computed tomography images were converted to grayscale and reviewed by an experienced radiologist. In order to make the CNN more efficient, for every different network used in the study, the input layer was replaced with a new one based on the size of COVID-19 infection patches and the dimensions of the last fully connected layer were set to the number of classes. The optimizer used was SGDM, the values of learning rate equals 0.01 and validation frequency was set to 5. The dataset is divided to 80\% train data and 20\% validation data. In each epoch the dataset was shaffled and if the training process stayed the same, the training process stopped.

\begin{table}[H]
  \begin{center}
    \caption{Results of 10 CNN}
    \label{tab:table1}
    \begin{tabular}{||c c c c c ||}
      \hline 
      \textbf{Reference} & \textbf{Network} & \textbf{Depth} & \textbf{Parameters} & \textbf{Accuracy}\\ 
      \hline
      \cite{alexnet} & AlexNet & 8 & 61 & 78.92  \\ % <--
      \hline
      \cite{vgg} & VGG-16 & 16 & 138 & 83.33  \\ % <--
	  \hline
      \cite{vgg} & VGG-19 & 19 & 144 & 85.29  \\ % <--
      \hline 
      \cite{squeezenet} & SqueezeNet & 18 & 1.24 & 82.84  \\ % <--
      \hline
      \cite{googlenet} & GoogleNet & 22 & 7 & 85.29  \\ % <--
      \hline
      \cite{mobilenet} & MobileNet-V2 & 53 & 3.5 & 92.16  \\ % <--
      \hline
      \cite{resnet} & ResNet-18 & 18 & 11.7 & 91.67  \\ % <--
      \hline
      \cite{resnet} & ResNet-50 & 50 & 25.6 & 94.12  \\ % <--
      \hline
      \cite{resnet} & ResNet-101 & 101 & 44.6 & 99.51  \\ % <--
      \hline
      \cite{xception} & Xception & 71 & 22.9 & 99.02  \\ % <--
      \hline
    \end{tabular}
  \end{center}
\end{table}

\textbf{Automated detection of COVID-19 cases using deep neural networks with X-ray images}\cite{secondreview}
\\ 
\\
The second paper called “Automated detection of COVID-19 cases using deep neural networks with X-ray images” trained a CNN in order to detect if a person is healthy or suffers from COVID-19 or normal pneumonia. On the paper were used two different datasets. The first one was called "A COVID-19 X-ray image database" which was developed by Cohen JP \cite{dataset1secondreview}. The dataset does not contain enough metadata referring to patients nevertheless, there were 125 positive with COVID-19 patients from whom 43 were female and 83 were male. Another information is that out of 26 patients the average age of them was 55 years. The second dataset that was used was the “ChestX-ray8 database” which was developed by Wang et al. \cite{dataset2secondreview}. This dataset contained X-ray images with healthy patients and patients with normal pneumonia however, it does not provide any metadata for the patients. The network had two different variants. The first one was able to detect whether or not the patients suffer from COVID-19. In order to train the network, from the second dataset, only the X-rays with healthy patients  were used, to help the network to classify a patient as healthy or COVID-19 positive. In the second one, the network was able to detect if a patient is healthy or suffers from COVID-19 or suffers from normal pneumonia. For this network, all two datasets were combined in order to get all three different results. The network used was the “DarkCovidNet” which is based on “Darknet-19” \cite{darknet}, the optimizer was Adam, cross entropy was used as a loss function and the learning rate was 3e-3. 


\begin{figure}[H]
\includegraphics[width=\linewidth]{assets/architecture-of-the-darkcovidnet-ozturk-tulin-et-al.png}
  \caption{Architecture of the DarkCovidNet}
  \label{fig:paper3architecture}
\end{figure}

\textbf{Computer-aided detection of COVID-19 from X-ray images using multi-CNN and Bayesnet classifier}\cite{thirdreview}

This paper has a combination of features extracted from multi-CNN and used the Bayesnet classifier for the prediction of COVID-19. The networks that used were Squeezenet\cite{squeezenet}, Darknet-53\cite{darknetpaper3}, MobilenetV2\cite{mobilenet}, Xception\cite{xception}, Shufflenet\cite{shufflenet} in order to produce a feature matrix of dimension 950x5000. Each network was pre-trained using Imagenet\cite{imagenet}. The feature matrix is passed to the Bayesnet classifier which classifies the images into COVID-19 and non-COVID categories. The first dataset is a combination of a dataset created by Cohen et al\cite{dataset1secondreview} and a dataset by Kaggle\cite{dataset1.2thirdreview} and has 453 COVID-19 images and 497 non-COVID images(bacterial, varial pneumonia) and had 91.16 percent accuracy. The second dataset\cite{dataset2thirdreview} had 71 COVID-19 images and 7 non-COVID images and had 97.44 percent accuracy. 

\begin{figure}[H]
\includegraphics[width=\linewidth]{assets/architecture-of-the-proposed-method-abraham-bejoy-et-al.png}
  \caption{Architecture of the proposed method}
  \label{fig:paper3architecture}
\end{figure}


\section{Results}

\section{Research}

\section{Results}

\section{Future Work}

\begin{thebibliography}{9}
\bibitem{firstcases} 
Zhu, Na, et al. 
\textit{A novel coronavirus from patients with pneumonia in China, 2019.}. 
New England Journal of Medicine (2020).

\bibitem{symptoms1}
Li, Long‐quan, et al.
\textit{COVID‐19 patients' clinical characteristics, discharge rate, and fatality rate of meta‐analysis.}
Journal of medical virology 92.6 (2020): 577-583.

\bibitem{symptoms2} 
Huang, Chaolin, et al. 
\textit{Clinical features of patients infected with 2019 novel coronavirus in Wuhan, China.}.
The lancet 395.10223 (2020): 497-506.

\bibitem{patients}
Li, Bo, et al. 
\textit {Prevalence and impact of cardiovascular metabolic diseases on COVID-19 in China.} 
Clinical Research in Cardiology 109.5 (2020): 531-538.

\bibitem{medicalCOVID}
Abdulamir, Ahmed S., and Rand R. Hafidh.
\textit{The Possible Immunological Pathways for the Variable Immunopathogenesis of COVID--19 Infections among Healthy Adults, Elderly and Children.}
Electronic Journal of General Medicine 17.4 (2020).

\bibitem{firstreview}
Ardakani, Ali Abbasian, et al. 
\textit{Application of deep learning technique to manage COVID-19 in routine clinical practice using CT images: Results of 10 convolutional neural networks.}.
Computers in Biology and Medicine (2020): 103795.

\bibitem{alexnet}
Krizhevsky Ilya Sutskever, et al. 
\textit{Imagenet classification with deep convolutional neural networks.}. Advances in neural information processing systems. 2012.

\bibitem{vgg}
Simonyan, Karen, et al. 
\textit {Very deep convolutional networks for large-scale image recognition.}. arXiv preprint arXiv:1409.1556 (2014).

\bibitem{squeezenet}
Iandola, Forrest N., et al. 
\textit {SqueezeNet: AlexNet-level accuracy with 50x fewer parameters and< 0.5 MB model size.}. 
arXiv preprint arXiv:1602.07360 (2016).

\bibitem{googlenet}
C. Szegedy et al. 
\textit{Going deeper with convolutions}
2015 IEEE Conference on Computer Vision and Pattern Recognition (CVPR), Boston, MA, 2015, pp. 1-9, doi: 10.1109/CVPR.2015.7298594.

\bibitem{mobilenet}
M. Sandler, A. Howard et al. 
\textit{MobileNetV2: Inverted Residuals and Linear Bottlenecks}
2018 IEEE/CVF Conference on Computer Vision and Pattern Recognition, Salt Lake City, UT, 2018, pp. 4510-4520, doi: 10.1109/CVPR.2018.00474.

\bibitem{resnet}
K. He, X. Zhang, et al. 
\textit{Deep Residual Learning for Image Recognition.}. 
2016 IEEE Conference on Computer Vision and Pattern Recognition (CVPR), Las Vegas, NV, 2016, pp. 770-778, doi: 10.1109/CVPR.2016.90.

\bibitem{xception}
Chollet, François. 
\textit{Xception: Deep learning with depthwise separable convolutions.}. Proceedings of the IEEE conference on computer vision and pattern recognition. 2017.

\bibitem{secondreview}
Ozturk, Tulin, et al. 
\textit{Automated detection of COVID-19 cases using deep neural networks with X-ray images}. 
Computers in Biology and Medicine (2020): 103792.

\bibitem{darknet}
Redmon, Joseph et al.
\textit{YOLO9000: better, faster, stronger.}.
Proceedings of the IEEE conference on computer vision and pattern recognition. 2017.

\bibitem{dataset1secondreview}
J.P. Cohen
\textit{COVID-19 Image Data Collection. 2020.}
https://github.com/ieee8023/COVID-chestxray-dataset.

\bibitem{dataset2secondreview}
X. Wang et al. 
\textit{Chestx-ray8: hospital-scale chest x-ray database and benchmarks on weakly-supervised classification and localization of common thorax diseases}
Proceedings of the IEEE Conference on Computer Vision and Pattern Recognition, 2017, pp. 2097–2106.

\bibitem{thirdreview}
Abraham, Bejoy et al. 
\textit{Computer-aided detection of COVID-19 from X-ray images using multi-CNN and Bayesnet classifier.}.
Biocybernetics and biomedical engineering 40.4 (2020): 1436-1445.

\bibitem{dataset1.2thirdreview}
Kermany, et al.
\textit{Identifying medical diagnoses and treatable diseases by image-based deep learning.}.
Cell 172.5 (2018): 1122-1131.

\bibitem{dataset2thirdreview}
Dadario AMV. Covid-19 X rays; 2020.
\textit{http://dx.doi.org/10.34740/KAGGLE/DSV/101946}. 
Available from:https://www.kaggle.com/dsv/1019469. 

\bibitem{imagenet}
Deng J et al.
\textit{Imagenet: alarge-scale hierarchical image database.}.
2009 IEEEConference on Computer Vision and Pattern Recognition.IEEE; 2009. p. 248–55.

\bibitem{shufflenet}
Zhang X et al.
\textit{Shufflenet: an extremelyefficient convolutional neural network for mobile devices.}.
Proceedings of the IEEE Conference on Computer Visionand Pattern Recognition; 2018. p. 6848–56.

\bibitem{darknetpaper3}
Redmon J et al. 
\textit{Yolov3: an incremental improvement.}.
2018, arXiv preprint arXiv:1804.02767
\end{thebibliography}
\end{document}